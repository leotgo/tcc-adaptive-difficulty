\chapter{METHODOLOGY AND RESULTS}
\label{ch:methodology-results}

\section{Overview}

- Overview \\
- About the hypotheses and objectives \\
- About the user selection criteria \\
- About the classification of the user base \\
- About the experiment 1 \\
- About the experiment 2 \\
- About the analysis of results 

% ======================================================================
% ======================================================================
% ======================================================================

\section{Hypotheses and General Objectives}

% * Hypotheses
% * ====================================

% Main Objectives
% ===================
%       - Verify if dynamic adjustments create an increase in the performance over the play session for beginner & intermediate players
%       - Verify if dynamic adjustments keep a fair challenge, without overt increases in performance for advanced players
As our general objectives when performing this experiment, we wish to verify if dynamic adjustments can create a more significative increase in performance in the first and second playthroughs for players that are inexperienced with the \emph{Souls-like} or action genres. We attempt to understand if players that experience the dynamic difficulty implementation first achieve a better understanding of the game mechanics and systems on their second playthrough.

We also wish to understand if dynamic difficulty can maintain a fair level of challenge to veterans or players that experience the game through a second playthrough. To do this, we wish to evaluate the difference in performance between fixed and dynamic difficulty on a second playthrough where players have likely increased their performance when using game mechanics. Based on these objectives, we formulate the following hypotheses:
% Hypothesis 1: Dynamic can alleviate the learning curve for beginners
% Hypothesis 2: Dynamic is less frustrating than fixed for beginners
% Hypothesis 3: Dynamic can keep a better challenge on 2nd+ playthrough
% Hypothesis 4: Dynamic is not overtly easy to veterans
\begin{itemize}
    \item{Hypothesis 1: Dynamic difficulty can alleviate the steep learning curve of challenging games for inexperienced players;}
    \item{Hypothesis 2: Dynamic difficulty can keep a better challenge on the second or subsequent playthroughs.}
\end{itemize}

% ======================================================================

% * Scenarios
% * ====================================

% * Variables
% * ====================================

% ======================================================================
% ======================================================================
% ======================================================================

\section{User Selection Criteria}
% * User selection criteria
% * ====================================

\sepfootnotecontent{fn:bloodborne}{Footnote: What is Bloodborne}
\sepfootnotecontent{fn:sekiro}{Footnote: What is Sekiro}
\sepfootnotecontent{fn:steam}{Footnote: What is Steam}

The methodology used for our experiment involves the deployment and use of our application in the environments which are most commonly used by each user to play games on, without in-person monitoring by the experiment observers. For the analysis of user performance and results, we make use of remotely collected data through telemetry techniques and compare it to user classification and perception surveys. However, there is still the issue of guaranteeing that users will use the application with similar environmental conditions. Therefore, we enforce selection methods for our user base to ensure similar characteristics on how the application is used.

The selection of users for our experiment uses a general set of characteristics and restrictions to eliminate the possibility of factors such as personal preferences, age, platform or familiarity with input devices affecting the performance and results of users. The selection of users was performed on multiple online gaming communities using the \emph{Discord} app. The targeted groups included general gaming communities that encompassed players of all skill levels and game genres, and communities focused at discussing the \emph{Souls-like} games such as \emph{Dark Souls}, \emph{Bloodborne} and \emph{Sekiro}. We define in the list below general player profile characteristics for user selection in our experiment:
% Player profile
% ===================
%   - Age: 18-23
%   - Plays Console or PC video games
%   - Prefers the use of joystick controllers
%   - Has a Steam account
\begin{itemize}
    \item{\emph{Age:} between 18 and 23 years old;}
    \item{\emph{Preffered gaming platforms:} Video Game Consoles or Desktop PC;}
    \item{\emph{Preffered input methods:} dual-analog joysticks;}
    \item{\emph{Additional requirements:} has a Steam\sepfootnote{fn:steam} account.}
\end{itemize}

% \subsection{Hardware Requirements}
% Hardware requirements
% ===================
We also require users to possess hardware that satisfies the proper execution of our application under a target framerate of 60 frames per second, to ensure that player performance is not affected by low graphical performance, stuttering or input lag. In the list below we specify the hardware requirements to properly execute our application under our specified hardware performance targets:
% Hardware Requirements \\
% 64 bit processor and Operating System \\
% Processor: Intel Core i5-2300 2.8GHz+ or AMD FX-6300 3.5GHz+ \\
% Graphics: NVIDIA GeForce GTX 460 or AMD Radeon 5000 series+ \\
% DirectX: Version 11+ \\
% Operating System: Windows 7 64bit, Service Pack 1+ \\
% Memory: 6GB+ RAM \\
% Available Storage: 4GB+ available on HDD or SSD \\
% 16:9 or 16:10 monitor between 19" and 24" and at least 1280x720 native resolution \\
\begin{itemize}
    \item{\emph{System Architecture:} 64 bit processor and Operating System;}
    \item{\emph{Processor:} Intel Core i5-2300 2.8GHz+ or AMD FX-6300 3.5GHz+;}
    \item{\emph{Graphics:} NVIDIA GeForce GTX 460 or AMD Radeon 5000 series+;}
    \item{\emph{DirectX:} Version 11+;}
    \item{\emph{Operating System:} Windows 7 64bit, Service Pack 1+}
    \item{\emph{Memory:} 6GB+ RAM}
    \item{\emph{Available Storage:} 4GB+ available on HDD or SSD}
    \item{\emph{Display:} 16:9 or 16:10 monitor sized between 19" and 24" and at least 1280x720 native resolution.}
\end{itemize}

% \subsection{Required Peripherals \& Input Methods}
We also require that players use the same input interface methods, to ensure that the accuracy of performed actions such as character movement, camera movement, dodging or blocking are not affected by restrictions in input devices, such as a computer keyboard not being able to represent directional movement with the same level of freedom as a dual-analog joystick. In the following list we specify the characteristics used to define the input interfaces required to be used by players in our experiment:  
% Required Peripherals & Input
% ===================
% Required peripherals and input: \\
%  - Joystick controller:  \\
%      - Xbox Controllers: Xbox 360, Xbox One, Xbox Series X \\
%      - Playstation Controllers: DualShock 3, Dual Shock 4, Dual Sense \\
%      - Or equivalent models with at least: \\
%          - Dual analogs \\
%          - 4 face buttons \\
%          - 4 digital directional buttons \\
%          - Right and left triggers and bumpers
\begin{itemize}
    \item{\emph{Xbox Controllers:} Xbox 360, Xbox One, Xbox Series X;}
    \item{\emph{PlayStation Controllers:} DualShock 3, DualShock 4, Dual Sense;}
    \item{\emph{Equivalent Joystick Controller Models} with at least:}
    \begin{itemize}
        \item{Dual Analog axes, one at each side;}
        \item{A clickable Trigger Button at each analog axis;}
        \item{4 Face Buttons in the right side;}
        \item{4 Digital Directional Buttons in the left side;}
        \item{Right and Left Triggers and Bumpers at the top.}
    \end{itemize}
\end{itemize}

We apply our user selection criteria through a Player Classification Survey, which is further detailed in section \ref{sec:user-base-classification} and is also used to subdivide our users in skill-based classification groups for \emph{Souls-like} games, such as \emph{Beginner}, \emph{Intermediate} and \emph{Veteran} users. Details on specific questions and metrics used in the survey can be seen in annex \ref{anx:player-classification-survey}.

% ======================================================================
% ======================================================================
% ======================================================================

\section{Classification of the User Base}
\label{sec:user-base-classification}
% * User classification groups
% * ====================================

% Objective
% ===================
%   - Separate players in beginners, intermediate, veterans
%       - Beginners: casuals, unused to action games
%       - Intermediate: average, used to action games
%       - Veterans:  hardcore, used to action & Souls-like games

% Metrics
% ===================
%   - Avg. Playtime per Week
%   - Experience in action games
%   - Completion of Souls-likes

% Classification Groups
% ===================
%   - Beginners:
%       - Never played a souls-like
%       - Low experience in action games
%       - Low avg. playtime
%   - Intermediate
%       - Previously played souls-likes
%       - Never finished a souls-like
%       - Medium or high experience in action games
%       - Moderate or high avg. playtime
%   - Veterans
%       - Finished one or multiple souls-likes
%       - High experience in action games
%       - High avg. playtime

% ======================================================================
% ======================================================================
% ======================================================================

% \section{Experiment 1: Validation of Simplified Object of Study}
% * Experiment 1: Validation of Simplified Object of Study
% * ====================================

% \subsection{Methodology}
% * Methodology
% * =======================

%\subsubsection{Restrictions}

% Restrictions
% ===================
%   - No dynamic adjustments
%   - Only advanced difficulty
%   - 3 levels

% \subsubsection{Goals}

% Goals
% ===================
%   - Verify if gameplay, aesthetics and challenge meet player expectations for a Dark Souls replica
%   - Observe performance metrics & define adjustment thresholds

% \subsubsection{User Base Description}

% User base
% ===================
%   - Total number: 10
%       - Beginners: 2
%       - Intermediates: 2
%       - Veterans: 6

% \subsubsection{Metrics}

% Metrics
% ===================
%   - Implementation validation survey
%       - Perception on Gameplay Fidelity
%       - Perception on Challenge Level and Difficulty
%       - Perception on Aesthetic Fidelity
%       - Performance of User Groups

% \subsubsection{Implementation Validation Survey Details}

% Implementation Validation Survey Details
% ===================

% \subsubsection{Experiment Flow}

% Experiment overview
% ===================
%   - Step 1: Perform player classification survey
%   - Step 2: Play through 3 levels with advanced difficulty
%   - Step 3: Perform implementation validation survey

% \subsection{Results}
% * Results
% * =======================

% \subsubsection{Perceptions on Gameplay \& Aesthetical Fidelity}

% \subsubsection{Perceptions on Challenge Level and Difficulty}

% \subsubsection{Performance of Player Skill Levels}

% Figures:
%   - One graph Per Player Classification Type

% Negative Metrics (lower is better)
%   - Avg. Completion Time per Level
%   - Avg. Number of Deaths Per Level
%   - Avg. Health Lost Per Encounter

% Positive Metrics (higher is better)
%   - Avg. Attack Avoidance Efficiency
%   - Avg. Attack Window Efficiency
%   - ? Avg. Damage Dealt Per 10s In Encounters
%   - ? Avg. Stamina Level In Encounters

% ======================================================================
% ======================================================================
% ======================================================================

\section{Methodology}
% * Experiment 2: Validation of Adaptive Solution
% * ====================================
\label{sec:experiment-methodology}

% * Methodology
% * =======================

\subsection{Goals}

% Goals
% ===================
%   - Verify if dynamic adjustments create an increase in the performance over the play session for beginner & intermediate players
%   - Verify if dynamic adjustments keep a fair challenge, without overt increases in performance for advanced players

\subsection{User base}

% User base
% ===================
%   - Total number: 23
%       - Beginners: 10
%       - Intermediates: 7
%       - Veterans: 6

% Division of user groups
% ===================
%   - Group A: Fixed first, adaptive second
%       - Segment 1: Fixed difficulty modes
%       - Segment 2: N-dimensional adaptive difficulty
%   - Group B: Adaptive first, fixed second
%       - Segment 1: N-dimensional adaptive difficulty
%       - Segment 2: Fixed difficulty modes

% - Separation of groups is done because we wish to evaluate user perception of difficulty comparing both versions
% - Users tend to have an easier time with the second version, as they had more time to learn how to play
% - 

% Number in user groups
% ===================
%   - Group A: 12
%       - Beginners: 5
%       - Intermediates: 4
%       - Veterans: 3
%   - Group B: 11
%       - Beginners:  5
%       - Intermediates: 3
%       - Veterans: 3

\subsection{Metrics}

% Metrics
% ===================
%   - Player Adaptive System Performance Metrics
%   - Player Perception & Comparison Survey

\subsection{Player Perception Survey Details}
% Player Perception & Comparison Survey
% ===================

\subsection{Experiment Flow}

% Experiment overview
% ===================
%   - Step 1: Perform player classification survey
%       - Be classified as beginner, intermediate, 
%   - Step 2: Be assigned to Group A or Group B
%   - Step 3: Play through part 1
%   - Step 4: Play through part 2
%   - Step 5: Perform player perception & comparison survey

\section{Results}
% * Results
% * =======================

% Figures:
%   - One graph Per Player Classification Type
%   - Bar Charts Comparing Fixed vs Dynamic Difficulty

% Negative Metrics (lower is better)
%   - Avg. Completion Time per Level
%   - Avg. Number of Deaths Per Level
%   - Avg. Health Lost Per Encounter

% Positive Metrics (higher is better)
%   - Avg. Attack Avoidance Efficiency
%   - Avg. Attack Window Efficiency
%   - ? Avg. Damage Dealt Per 10s In Encounters
%   - ? Avg. Stamina Level In Encounters


\subsection{Perceptions on Part 1 vs. Part 2}

% Difficulty
- Perception Survey answers regarding difficulty of part 1 vs part 2

% Learning Curve
- Perception Survey answers regarding learning curve of part 1 vs part 2

\subsection{User Group Performance in Part 1}

% Figures:
%   - Comparison Group A Part 1 (Fixed) vs Group B Part 1 (Dynamic)

\subsection{User Group Performance in Part 2}

% Figures:
%   - Comparison Group A Part 1 (Fixed) vs Group B Part 1 (Dynamic)

% ======================================================================
% ======================================================================
% ======================================================================

\section{Conclusions}

\subsection{Summary of Results}
% * Summary of Results
% * =======================

\subsection{Limitations}
% * Limitations
% * =======================

\subsection{Comparison with Previous Methodologies}
% * Comparison with previous work
% * =======================

% ======================================================================
% ======================================================================
% ======================================================================