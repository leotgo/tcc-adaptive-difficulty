\sepfootnotecontent{fn:triple-a}{AAA (pronounced "triple A") games are an informal classification of video games distributed by major publishers and distinguished by large development and marketing budgets. In general, the term is used to represent the games that are supposed to achieve the highest standard of quality and content in video games.}

\sepfootnotecontent{fn:genshin-impact}{Genshin Impact (miHoYo, 2020). Video Game. Android, iOS, Windows, PlayStation 4, Playstation 5, Nintendo Switch.}

\sepfootnotecontent{fn:escape-tarkov}{Escape from Tarkov (Battlestate Games, 2017). Computer Game. Microsoft Windows.}

\sepfootnotecontent{fn:hollow-knight}{Hollow Knight (Team Cherry, 2017). Video Game. PlayStation 4, Nintendo Switch, Xbox One, macOS, Linux, Microsoft Windows.}

\sepfootnotecontent{fn:ghost-tale}{Ghost of a Tale (SeithCG \& Plug In Digital, 2016). Video Game. PlayStation 4, Nintendo Switch, Microsoft Windows, Xbox One.}

\sepfootnotecontent{fn:cuphead}{Cuphead (Studio MDHR Entertainment Inc., 2017). Video Game. PlayStation 4, Nintendo Switch, Xbox One, Microsoft Windows, macOS.}

\sepfootnotecontent{fn:undead}{\emph{Undead} beings are fictional characters represented by entities that were reanimated from death through supernatural means. Common representations of \emph{Undead} beings in popular media include \emph{Zombies}, \emph{Skeletons} and \emph{Vampires}.}

\sepfootnotecontent{fn:particle-vfx}{\emph{Particle systems} in video games are systems which manipulate 2D sprites or simple 3D models to simulate natural phenomena, such as fluids or chemical reactions. They are often used in real-time 3D rendering applications as a cheap alternative to simulating fluid physics.}

\sepfootnotecontent{fn:death-stranding}{Death Stranding (Kojima Productions, 2019). Video Game. PlayStation 4, PlayStation 5, Microsoft Windows.}

\sepfootnotecontent{fn:core-gameplay-loop}{A \emph{Core Gameplay Loop} is an abstraction that describes the repeated cycle of activities performed by the player in a game. Core gameplay loops are used to represent the simplest and most essential activities that constitute player interaction and objectives in a game, and serve as a guideline for game designers to develop game mechanics and dynamics.}

\sepfootnotecontent{fn:character-controller}{\emph{CharacterControllers} are system components which contain functionality to move a player-controlled character in a two-dimensional or three-dimensional environment. In Unity3D, the CharacterController component contains reusable functionality for handling collision with the ground and walls, moving over slopes and climbing over steps.}

\sepfootnotecontent{fn:movement-drifting}{\emph{Movement drifting} occurs when the position translation of a game entity does not reflect the visual representations of their movement. It is a common issue in game animation, where a player's speed does not accurately reflect the animation performed by the character model.}

\sepfootnotecontent{fn:lock-on}{Lock-On is a common mechanic in games with considerable freedom of movement to lock the player character's aim to a target. The mechanic facilitates landing successful strikes at enemies, and in third person games allows \emph{circular strafing} around the target.}

\sepfootnotecontent{fn:colliders}{\emph{Colliders} are simplified and invisible meshes used to perform collision intersection checks in physics-based simulations. Colliders are widely used in games due to their low computational cost, as well as their native support in game engines.}

\sepfootnotecontent{fn:collider-overlapping}{\emph{Collider overlapping}, also known as \emph{Collision detection} or \emph{Collision intersection checking}, is the act of detecting an intersection between two geometry shapes. In 3D video games, collision checking often uses simplified geometry shapes such as cuboids and spheres for more efficient resolution algorithms which can attend to the performance requirements of the application.}

\sepfootnotecontent{fn:monobehaviour}{\emph{MonoBehaviour} is the base class which every script derives from in Unity. MonoBehaviours are components which can be attached to GameObjects to provide reusable functionality. Any objects instanced in game levels that contain any customizable behavior will necessarily contain a MonoBehaviour component.}

\sepfootnotecontent{fn:fixed-update}{The \emph{FixedUpdate} is a frame-rate independent function commonly used in game engines to implement functionality regarding physics systems, such as player movement. The update occurs regardless of stutters or delays caused by the rendering system. This characteristic is required for physics simulation to be consistent regardless of hardware-related issues.}

\sepfootnotecontent{fn:rays}{\emph{Rays} are commonly used in video games to represent three-dimensional non-volumetric vectors with an origin position, a length and a direction. Such vectors can be used to perform collision intersection checks for specific game systems which do not involve renderable game entities. For instance, in first-person shooter games, rays are commonly used to check whether a player succesfully shoots a target centered at their \emph{Crosshair}.}

\sepfootnotecontent{fn:raycasts}{\emph{Ray Casting} is a collision intersection resolution algorithm where a \emph{ray} is projected and tested against colliders in a 3D environment. In the context of our implementation, Raycasting is used to detect collision of entities which move at high speeds, such as the player being affected by gravity.}

\sepfootnotecontent{fn:mecanim}{\emph{Mecanim} is Unity's animation system, which defines rules for the hierarchy of humanoid \emph{Skeletal Rigs} and a relationship between \emph{Animation States} through \emph{Finite State Machines}. Mecanim provides a visual editor to configure animation rigging and animation state relationships, and an API which can be accessed by \emph{MonoBehaviours} to update animation-related parameters.}

\sepfootnotecontent{fn:skeletal-rigs}{\emph{Skeleton-based rigging} is a technique used in computer graphics to manipulate groups of vertices of a 3D model as an hierarchical set of interconnected parts. Rotations performed in parent nodes of the hierarchy will result in rotations in child nodes. This technique is commonly used to animate organic figures, and the animation process is intuitively simpler than manually manipulating vertices.}

\sepfootnotecontent{fn:motion-sickness}{\emph{Motion sickness} is a sick feeling which occurs due to a difference between real and expected movement. In video games, motion sickness is a common issue inherent to camera movement algorithms in first-person games.}

\sepfootnotecontent{fn:unity-prefabs}{\emph{Prefabs} in Unity are pre-configured hierarchies of reusable \emph{GameObjects} which can be instanced in multiple game levels. Examples of Prefabs in our implementation include the Player character, the application's UI and modular 3D environment pieces which are assembled into levels.}

\sepfootnotecontent{fn:unity-gameobjects}{\emph{GameObjects} in Unity represent every entity which can be instanced in a game level, and which might contain rendering, audio, systems or behavior-related components.}

\sepfootnotecontent{fn:transposing-and-composing}{Transposing and Composing strategies are used by the \emph{Cinemachine} extension to determine how a Camera is placed in a scene, and how a shot is framed.}

\sepfootnotecontent{fn:observer-pattern}{The \emph{Observer pattern} is a behavioral software design pattern which defines a publish-subscribe relationship between an object (the Subject) and its dependents (the Observers). It is mainly used to decouple systems and create a flexible layer for one-to-many dependencies.}

\sepfootnotecontent{fn:bounding-boxes}{\emph{Bounding boxes} are cuboid volumes which completely envelop an object in a two-dimensional or three-dimensional space. In games, bounding boxes are tipically used for simplified collision intersection checking, as they are sufficiently precise and less expensive than calculating the collision for complex meshes.}

\sepfootnotecontent{fn:stagger}{A \emph{Stagger} effect renders a character unable to perform any actions for a short amount of time. In most video games, functionality for a Stagger is the same as a \emph{Stun} effect, but the effect occurs throughout a much shorter length, while also displaying a different and unique animation to the player.}

\sepfootnotecontent{fn:knockback}{A \emph{Knockback} effect physically pushes a character away from a source of impact. In our implementation, the target of the effect is unable to perform any actions while being pushed.}

\sepfootnotecontent{fn:cooldown}{A \emph{Cooldown} timer represents the fixed amount of time where a specific action is blocked to an entity after performing it. In the case of AI agents in our implementation, enemies are unable to attack for a fixed time interval after a previous attack took place.}

\sepfootnotecontent{fn:forward-vector}{The \emph{forward} vector represents the direction of the frontal normal of a 3D object when considering its bounding box. Game entities are considered to "face" a direction when their forward vector points towards such direction.}

\sepfootnotecontent{fn:poise-break}{A \emph{Poise Break} occurs when a character's poise attributes temporarily reaches a value of zero. In such situtation, the character is briefly \emph{Staggered}, and becomes vulnerable for enemy attacks.}