\chapter{ANALYSIS OF OBJECT OF STUDY}
\label{ch:analysis-object-study}
\label{sec:analysis-dark-souls}

% Justify choice of dark souls
% Overview of mechanics
% Identify features for implementation
% Analyze AI
In this section, we justify the motivation behind the choice of Dark Souls as our object of study. We perform an overview of the most relevant combat mechanics to identify the subset of features that should be implemented in our simplified version. We identify the core aspects of the implementation of AI agents in Dark Souls, defining a set of guidelines to aim for when attempting to replicate the behavior of enemies in our implementation.

% Analyze Level Design
% Analyze Difficulty Factors
% Summarize observations, identify issues and propose solutions
We perform a generalized analysis of the philosophy applied in level design, to understand which methodologies were applied and which of its aspects are difficulty factors to players. We identify the difficulty factors in Dark Souls, attempting to understand the inherent issues with its design decisions and systems. We conclude by summarizing the observations and analysis performed in this section to define the aspects that our simplified implementation should attempt to replicate, as well as pointing out the issues regarding difficulty and approachability that can be addressed through an adaptive solution.

% ========================================================================
% ========================================================================
% ========================================================================

\section{Motivation of Choice}

% Dark souls pushed the extents of difficulty in games
Dark Souls has successfully pushed the extents of challenge in Action RPGs to their limit, providing combat with a simple yet carefully designed set of core mechanics. However, the greatest capability of the \emph{Dark Souls} series, the design of its difficulty, is for some an unwelcome trait. While the game presents enemies with simple pattern-based AI, players will be firmly punished for their mistakes. For this reason, \emph{Dark Souls} is often used as a reference in discussions regarding difficulty in games \cite{URL_ExploringDesignOfDarkSouls}.

% Death is used as a learning mechanic
In \emph{Dark Souls} the player does not fail by dying, but by giving up on playing the game altogether. The game design creates a cycle of experimentation, learning, adaptation and, after multiple iterations, progression through the levels. Death is not considered failure, but simply a means to attain knowledge and training. Because of this type of punishing yet rewarding design which teaches the player through death, the special edition of the first title in the \emph{Dark Souls} series yields the subtitle "Prepare to Die Edition".

% Dark Souls is rewarding even with seemingly simple challenges
While the average action game might reward the player for beating entire armies of enemies on his own, \emph{Dark Souls} can be a rewarding experience for the player even when seemingly basic encounters are surpassed, such as surviving a battle with a single enemy. When adventuring through the tight corridors of a dungeon or dark underground caves, the player faces unknown dangers and must fight their way out. Even the weakest enemies can surprise and kill an unsuspecting player. When facing the strong opponents, dealing the finishing blow in an intense fight means the player overcame a challenge. Understanding the design of Dark Souls is, therefore, a key component to understanding the relationship of challenge and reward.

% ============================================================================
% ============================================================================
% ============================================================================

\section{Overview of Game Mechanics}

% ============================================
% Cameras
% ============================================

% Dark Souls is Third Person
Dark Souls is for the most part presented in a Third Person Camera perspective, with the player character positioned close to the center of the screen. The Third Person perspective is distinguished from a First-Person perspective by the visibility of the player character's model and their surroundings. Third person perspectives are optimal for action games with close-quarters combat because they provide a broad view of the the environment, enemies and their distance to the player.

% What is Orbital TP Camera
In Orbital Third Person Cameras such as the implementation in Dark Souls, character movement is based on camera orientation \cite{BOOK_RealTimeCameras}. If the player moves forward, their controlled character moves towards the direction the camera is facing. The camera is also constrained to rotating around the player in a fixed distance, making their movement resemble an orbit.

% Camera pulled forward to be maintained inside playable space
However, there are cases in which maintaining a fixed distance to the player is impossible, such as when the target position is outside the playable space. In such situations, the camera repositions inside the playable space whenever its target position would be out of boundaries. In Dark Souls the target position is calculated by simply pulling the camera closer to the player character. The relative size of the player in screen is then greatly increased and might disrupt the visualization of important gameplay elements such as enemies or level geometry. 

% Obstacles might also impair player ability to see. Lock-On helps.
Environment elements such as columns, rocks or even non-playable characters might also impair the visualization of the player character and non-player characters, thus hindering the ability for the player to visualize and assess the situation in a combat encounter. In response to these difficulties, Third Person games with close-quarters combat often implement a secondary "Lock-On" camera as an alternative to the default.

% What is Lock-On
Lock-On Cameras override the positioning and framing of Third Person Cameras by locking the orientation of the view to a "lock target". Instead of centering on the player character, the Lock-On Camera focuses on another object or position such as an enemy character, while still offsetting its position to frame the player character in view. This allows for optimal visualization of essential elements in a combat encounter. In this mode, player movement changes to a circular strafe method that is relative to the lock target's position. Moving sideways results in an arc-shaped motion around the target. Forward and backwards movement transposes the player character closer to or away from the target.

% Lock-On cameras are good for dodging sideways, adjusting distance, aiming attacks
The Lock-On Camera is ideal for single-target combat situations, as the player can avoid frontal enemy attacks by moving sideways, while still being able to manipulate their  distance for counterattacks. In the case of enemies using Area of Effect attacks which target circular area around them, the player can easily move away from the enemy. Additionally, player orientation is updated to constantly face the target, which simplifies the process of angling attacks towards enemies. Since the player character is constantly facing the direction of the target, successfully registering an attack simply requires the player to adjust their distance to the target.

% ============================================
% Combat
% ============================================

% Combat is slow, focus on strategy and precision. Encourage player to block and dodge.
Dark Souls is an Action RPG with heavy focus on close-quarters combat. In contrast to other games in the genre, is not fast-paced or dynamic. Instead, the game focuses on encouraging the player to perform precise and strategic movements. Thus, Dark Souls leverages the characteristics of predecessors such as \emph{King's Field} by creating a sense of weight and impact on attacks, emphasizing the necessity of players to protect by dodging or blocking \cite{BOOK_DarkSoulsBeyondTheGrave}.

% How attacking/defending input works
The controls of games in the \emph{Souls} series operate in similar fashion: each of the character's arms are controlled by separate buttons, represented by the left and right shoulder pad in a Joystick. The player can wield different one-handed equipments in each arm, or a two-handed equipment in both arms. The character will perform actions for an arm based on which equipment is equipped, such as slashing with a sword or defending with a shield. If a two-handed weapon is equipped, using the right shoulder pad button will attack and the left will defend. The most common configuration is to have a shield in the left hand, and a weapon in the right.

% Multiple weapons, each weapon has its unique attacks
% Attacks use stamina, deal some damage, have recover time
% Light attacks vs heavy attacks
% Attacks can stagger enemies
During play-through, the player will come across a wide variety of weapons with multiple unique attacks.  Each attack requires a certain amount of \emph{Stamina}, deals a certain amount of \emph{Damage} when hitting an enemy, and has a \emph{Recover Time}. Certain attacks also have the chance of \emph{Staggering} an enemy. Heavy Attacks are slower and cost the most Stamina, but have a higher chance of Staggering an enemy. Light attacks are faster and can be linked for a quick succession of attacks, dealing significant damage but have a low chance of causing Stagger. These core mechanics create opportunities for strategic decisions that the player must take before and during combat. We list these mechanics, along with a detailed description of their in-game function:

\begin{itemize}

\item \emph{Damage}: the amount of health a character loses upon getting struck by an attack. This amount can be reduced or fully negated by blocking or dodging incoming attacks.

\item \emph{Health}: an attribute representing the overall physical state of a character. Numerically, it determines how much Damage a character can sustain before being destroyed.

\item \emph{Stamina}: an attribute that represents the character tiredness. Numerically, it determines the ability to perform actions in combat. Attacking, dodging and even shielding oneself against enemy attacks all have a \emph{Stamina} cost. Although \emph{Stamina} is a finite resource, it replenishes passively after the player stops performing combat actions.

\item \emph{Recover Time}: the amount of frames the player is unable to do any actions and is vulnerable to attacks after their own attack has taken place. This attribute is used to create a feeling of weight in attacks, and can be used by the player to punish slow enemies after missing an attack.

\item \emph{Stagger}: the condition where a character is unable to perform any actions and is vulnerable to attacks after getting struck by a powerful attack or a quick succession of attacks. In \emph{Dark Souls}, this condition is especially cruel to the player, since after losing a significant amount of health the player is still susceptible to a follow-up attacks, without the ability to defend their self.

\end{itemize}

% ============================================
% Defensive Actions
% ============================================

% Player can block, dodge, parry, move away
% How blocking works
To avoid getting struck by enemy attacks the player can Block, Dodge, Parry or simply move away from the attack. Out of the available defensive actions, Blocking is the safest and least skill dependent. A block reduces or completely eliminates the incoming damage of an enemy attack. While blocking, the player will replenish Stamina at a reduced rate. Each blocked attack has a Stamina cost based on its power and player attributes. If an enemy attack is too powerful and the player does not have the necessary Stamina or attributes to block, the player will be Staggered.

% How dodging works
Dodging can be performed by performing the "roll" action in an user-defined direction when an enemy is about to land an attack. While it costs Stamina, succeeding in this action completely negates the damage of an attack, regardless of how powerful the attack is. Upon failing, the player will take full damage. Thus, to avoid the negative effects of failure, dodging should be performed with care as it heavily depends on timing and accuracy.

% How parrying works
Parrying can be performed with precise timing to deflect an enemy attack and destabilize the enemy. The player can then perform a riposte attack, a powerful move that adds a considerable amount of damage in comparison to normal attack. The time window to successfully perform a parry is considerably smaller than that of dodging. Failing a Parry will cause the player to take full damage of the attack and be Staggered. Thus, Parrying is a high-risk high-reward action, and can also be considered an offensive move.

% Players should consider which defensive action to take
Blocking, Parrying and Dodging require the use of Stamina, an important resource which is also necessary to perform offensive moves. If a player decides to block all incoming attacks, their Stamina will not replenish quickly enough to sustain their defense indefinitely, which will eventually cause their character to be Staggered. When against multiple enemies, blocking can quickly deplete player Stamina. Dodging and Parrying are timing dependent, and consequently prone to mistakes and heavy punishment. If players repeatedly try to perform the same defensive action, they will often find themselves cornered, without Stamina and vulnerable to fatal blows. Experienced players should have the knowledge of when to performe each of these actions, or even when to perform none and simply move away to avoid attacks.

% ============================================================================
% ============================================================================
% ============================================================================

\section{Artificial Intelligence}

% No official information on Dark Souls AI
% It is possible to observe and replicate the AI of Dark Souls
During the literature review process conducted in this work, no official information regarding the implementation of AI in Dark Souls was discovered. According to \citeonline{YT_DarkSoulsSimpleAI}, some sources indicate a \emph{Hierarchical Task Planning Network System}, but an analysis of the in-game enemy actions provides no support to this claim. However, by performing multiple tests and playthroughs it is possible to reverse engineer the behavior of enemies and suggest a replication methodology that could satisfy the same properties of the original game.

% NPCs are all similar
% When detect player, attack from range or move closer
NPCs (Non-Player Characters) often share similar behavioral patterns in Dark Souls. While the specific actions of a humanoid NPC might differ from a quadruped, their overall behavior upon detecting player presence is the same. NPCs will stand idle until receiving information from their sensors, such as the player stepping into their line of sight. At that point, the enemy will either attack at range, if using a bow or spell, or move towards the player and attack in close-quarters.

% Enemy actions might vary depending on equipment or status
Enemy actions might slightly vary depending on their equipment or status. Enemies wearing a shield will commonly attempt to defend themselves when the player repeatedly attacks. The defense stance can be punished by the player by performing a kick or an attack from behind. A low health enemy might evade attacks or even heal itself. Most enemy actions are predictable, even when there is a certain variability in the patterns used.

% Predictability of AI can help player learn
% Dark Souls also creates overtly strong enemies
Ultimately, the AI of enemies in Dark Souls is simplistic but still sufficient to achieve the objectives proposed by the developer. The predictability of AI agents can be considered a favorable factor for the player when learning how to overcome an enemy. Still, this simplicity might not be perceivable by the player in their first playthrough, since they will simultaneously struggle with the inherent challenges of level design, such as level layout and 3D environment geometry. In addition, Dark Souls surprises new players by presenting overtly strong enemies in comparison to the player. A slight mistake might cause the player to endure punishing amounts of damage, which exacerbates the perception of how difficult an opponent is.

% Perception of AI smartness based on enemy health & damage
The perception of strength in Dark Souls's enemies carries similarities to what was reported during development of the original \emph{Halo} game by Bungie. Players perception of enemy smartness was skewed by how an enemy could sustain player damage and how much damage they dealt to the player \citeonline{URL_IllusionOfIntelligence}. Increasing Health points for an enemy type enhanced the perception of enemy intelligence for first-time playtesters. Thus, a perception of "toughness" could shadow the lack of intelligence of an enemy in a first playthrough. However, as the player repeatedly faces the same enemy upon countless deaths, the illusion gradually fades.

% ============================================================================
% ============================================================================
% ============================================================================

\section{Level Design}
\label{sec:level-design-dark-souls}

% What is level design
Level Design can be defined as the interpretation of Game Design and its application to the design of the challenges and situations that a player face over the course of a playthrough \cite{BOOK_LevelDesignConcept}. The Level Designer should understand the rules of a game and determine how the player is confronted by them. It can be argued that Game Design represents the theoretical definition of a game, whereas Level Design is the concrete implementation of Game Design.

% What is level design part 2
Level Design determines the layout of a playable environment, the placement of enemies, gameplay objects and the environmental hazards. In a sense, Level Design can be seen as a means of expressing Game Design through the creation of situations involving game entities and the playable environment. Therefore, a game developer must consider what experiences they intend for the player to engage in for a specific playable space.

\sepfootnotecontent{fn:esv-skyrim}{The Elder Scrolls V: Skyrim (Bethesda Game Studios, 2011). Computer Game. Microsoft Windows.}

% Map layout in Dark Souls
In the case of Dark Souls, the world takes place in an open, interconnected and vertically stacked map layout. When projecting this layout on a two-dimensional chart, it resembles the format of a spiral. This type of map layout is radically different than other games in the genre such as \emph{The Elder Scrolls V: Skyrim} \sepfootnote{fn:esv-skyrim}, which presents dungeon maps as horizontally spaced and linear paths.

% Characteristics of spiral level designs
Spiral level designs can feature a myriad of alternate routes and shortcuts. Since the player is constantly facing the danger of losing Experience Points upon dying, the Spiral layout creates opportunities for the player, with shorter paths that provide less risk when returning to safe zones. In the same philosophy, the game will often contain shortcuts from safe zones to later attained areas. These shortcuts must be unlocked by reaching a certain location and performing an action, such as finding the key to a locked gate. This Level Design technique is commonly known as \emph{gating} \cite{BOOK_LevelUpTheGuideToGreat}.

% Environment design & dangers in Dark Souls
By journeying through the world, the player will come across castles, dungeons, caves, and fortresses. It is more frequent to find oneself in small passageways than open areas, the latter being used more often in \emph{boss fight} areas. The need of constantly turning left and right, ascending stairs and descending through dark passages gives the developers numerous places to hide enemies and traps in. Thus, the player often faces threats such as being assaulted by an unseen enemy, getting hit by a trap or even falling to their death.  The careful placement of enemies, traps and pitfalls in creative fashion constitutes the core of \emph{Dark Souls} Level Design.

% All enemies can be potentially difficulty
% Dark Souls rewards patience, planning and careful execution
Each and every enemy the player faces has the potential to generate a difficult encounter, given their placement, the environment geometry and the others entities in an encounter. An enemy that is considered weak can still take away a considerable amount of Health from the player under some circumstances. However, most enemies are become vulnerable after performing an attack, which rewards player for being patient and attacking at optimal opportunities. Therefore, the player has the possibility of optimizing their play style by spotting enemies ahead of time, planning on how to exploit weaknesses and only then performing an action.

% Carefulness makes difficult encounters easier
This characteristic makes seemingly unfair encounters beatable. Groups can be separated into smaller sizes by luring enemies one by one. Traps can be used against the enemies, if the player lures the target to the area of effect. Enemies can be pulled from their territory into a safer and player-controlled position.

% Threats help with immersion and worldbuilding
However, the strength of a \emph{Souls}-like game isn't just its combat. The threatening aspect of in-game environments encourages the player to explore carefully, allowing more time to slowly experience the atmosphere and world lore.. The constant feeling of danger increases the tension and maintains the player aware of their situation to survive. According to \citeonline{YT_EvolutionOfDarkSoulsLevelDesign}, the Level Design in  Dark Souls is the main contributor to the player immersion in the game's narrative.

% ============================================================================
% ============================================================================
% ============================================================================

\section{The Difficulty Factors of Dark Souls}
\label{sec:pain-points-dark-souls}

% * Online: Can Approachability “Fix” Dark Souls?
% * ====================
% \cite{ONLINE_ApproachabilityFixDarkSouls}

We can begin to understand the difficulty factors of the game design in Dark Souls by analyzing game aspects through two different perspectives: that of beginner players, which are not accustomed to or skillful at playing action games; and that of veteran or highly skilled players, which are able to get acquainted with game mechanics and systems in a faster speed with less friction caused by repeatedly playing through the same game segments.

\subsection{Concept of Approachability}

% Approachability
%   - Concept explored in \cite{ONLINE_WhatAcessibilityMeansInGameDesign}
%   - Defines how feasible it is for a beginner or unskilled player to incrementally get accustomed to and learn game mechanics and systems
%   - Is about making changes to the gameplay and design of a title to allow more people to be able to play through it;
To analyze the perspective of beginner players when getting accustomed game mechanics and systems in difficult games, we explore the concept of \emph{approachability}, which is detailed in \cite{ONLINE_AcessibilityInGameDesign}. Approachability defines the ease at which a beginner or unskilled player is able to incrementally get accustomed to and improve upon the use of game mechanics and systems, and is focused on enabling an analysis of games that identifies changes in gameplay and design that might allow more people enjoy the process of learning and improvement.

%   - Directly tied to the concept of learning curve
%   - Affected by possible areas, features or systems that overwhelmingly difficult, and are stopping people from "getting into" the game
%   - Makes people part of the churn rates. 
Approachability is directly tied to the earlier discussed concept of learning curve, where the difficulty of a game must be balanced as to allow players to understand mechanics and systems in isolated environments, and in sequence improve their skills by playing through situations which integrate the use of such systems with the previously learned aspects of a game. Therefore, the approachability of a game is negatively affected by the areas, features and systems that might present an overwhelmingly difficult level of challenge in comparison to the skill of beginner players, which might cause beginner players to give up on playing a game.

%   - Approachability can be improved by having aspects that could be optionally tweaked to provide an easier experience, while still letting someone play through the game
Approachability in games can be improved by identifying the aspects of game design that are a detrimental factor on the balancing of the learning curves and challenge level of a game, and providing optional or automatic adjustments to the attributes that configure such aspects. The adjustments can be performed with to create an easier or more acceptable experience to new players, or to provide a more interesting challenge to veteran players or highly skilled players that could be experiencing another playthrough.

% Dark Souls Difficulty
%   - Difficulty is more about the player’s own abilities and skill set
%   - Skill ceiling doesn’t go that much higher than the skill floor.
Dark Souls is an especially interesting example of a difficult game for the analysis of  approachability issues, as the skill ceiling required for players to handle the difficulty curve is not greatly superior to the skill floor required from beginner players. This means that, when a player is able to achieve a certain level of competence in combat encounters, the contrast between challenge level and player skill is stabilized.

%   -Article in \cite{ONLINE_ApproachabilityFixDarkSouls} elicits some aspects of dark souls that affect approachability, and provides suggestions on possible fixes or attenuations that can be used.
Therefore, the main focus in the proposed challenges of Dark Souls is the development of player abilities and skill set, where players should learn how to best deal with each specific situation to overcome the proposed challenges, instead of the game providing challenges with a significant variation in difficulty level. The analysis performed in \cite{ONLINE_ApproachabilityFixDarkSouls} elicits some of the aspects of game design in Dark Souls that affect its approachability, relating each system to the set of skills required from the player and providing suggestions on possible fixes or features that can attenuate each issue.

\subsection{Detrimental Factors to Approachability in Dark Souls}
\label{sec:detrimental-factors-approachability}

% List of detrimental factors to approachability noticed in Bycer's article:
% =============================

% Problem: Game Speed
%   - Stumbling point for people not accustomed to action games or that react slowly;
%   - Involves reaction times, pattern recognition, decision making;
% Suggestions
%   - Option to slow down the combat speed would be helpful for approachability.
The first problem that was identified is regarding the speed of combat in Dark Souls, which affects the speed of attack animations and the frequency of actions performed by AI opponents. An accelerated combat speed can be one of the major stumbling points for players not accustomed to action games or that react slowly to in-game events, as it requires the player to respond wth accurate reaction timing, quick recognition of attack patterns and appropriate decision making. To attenuate the issues caused by the accelerated nature of combat speed in Dark Souls, the author proposes an accessibility option where players are able to parametrically slow down combat speed.

% Problem: Parrying
%   - One of the more advanced elements in any souls-likes;
%   - Involves reaction times; 
%   - In some of them, could be a handy feature to make combat easier;
%   - In others, it could be required for certain enemies and bosses to stand a chance;
% Suggestions
%   - Options to increase parry windows for beginners to get comfortable.
Another problem was identified involving the existence of the \emph{parrying} mechanic, where parrying efficiency is heavily tied to accurate and timing-based execution. A successful parry is able to completely negate the damage and effects caused to the player while also leaving enemies vulnerable, which is highly desirable. However, it can also bring significant prejudice to players if performed incorrectly. The author argues that the parrying mechanic is an useful feature to make combat easier in some action games, but it depends on the flexibility of its timing window.

In Dark Souls, parrying becomes a required mechanic to execute against some enemy types, as some of the attacks performed by enemies can not be dodged or defended against. Because of the brief timing window where the player is able to effectively parry an attack, the requirement of the parry mechanic in Dark Souls results in an increase of the difficulty level in some combat encounters, especially in the case of boss fights or when handling multiple enemies. The author suggests that the problems raised by the parrying mechanic in Dark Souls can be attenuated through accessibility options or alternate difficulty curves, where beginner players are able to increase parry windows to get comfortable with the parrying mechanic before partaking into the default level of challenge proposed by game designers.

% Problem: Combat Tells
%   - Player needs info on how to respond to attacks; 
%   - Involves player ability to recognize and respond to patterns;
The third problem identified was related to the nonexistence of visible indicators regarding the timing of enemy attacks, as well as a visual ambiguity in attack animations. In action games which involve combat, information regarding how players should respond to attacks should be communicated in a deterministic manner. Games such as Dark Souls employ multiple attack types, which can affect the player differently depending on their response. For instance, larger enemies will generally perform slower attacks that are easier to dodge, but if the player attempts to block such an attack a large amount of stamina will be spent and possibly cause the player to be staggered.

%   - Player should be able to use info to decide when to respond and what to do
%   - It is common for games to have tells for “unblockable” attacks;
Therefore, when an enemy is performing an attack the player should be able to identify and use the information communicated by game systems to decide which action to respond with and when to respond. For instance, it is a common trend in action games to provide a visual indicator when enemies are performing an \emph{unblockable} attack, commonly in the form of a visual effect which highlights the enemy character.

%   - Advanced players can attune themselves without alert systems; 
%   - A bad alert system can crush new players if not done right;
% Suggestions
%   - Player could get a visible indicator when the enemy is going to attack.
While veteran and highly skilled players can attune themselves to the nuances enemy animations in Dark Souls without the need of visible indicators, a beginner player tipically requires much more time and repeated attempts, which increase the initial friction. As a possible solution to this issue, and to provide more opportunities for beginner players to recognize such patterns, the authors suggest the option to include optional visual indicators when enemies are about to perform an attack, as well as specific visual effects depending on the type of attack being performed.  

% Problem: Healing
%   - Healing is a major aspect of the difficulty of in souls-likes;
Another difficulty factor was identified regarding resource sustaining through the correct use of healing items (which are named \emph{Eastus Flasks} in Dark Souls). The ability to heal affects the number of opportunities a player can experience to learn about enemies and combat situations. One of the key components of level design in souls-likes is the "exploration of the unknown", where players are constantly unaware of the challenges that will be faced at each point in a game -- which might render them unable to devise proper strategies to handle specific combat encounters. Therefore, one of the core aspects that affect player progression is their ability to sustain themselves while committing mistakes in unexplored combat encounters or environments.

%   - The more/faster you can heal the longer you can sustain yourself in a fight;
%   - Involves player ability to predict patterns and decision making;
In general, the larger the amount of health points a player is able to heal before reaching or re-spawning at a \emph{bonfire}, the longer the player is able to explore game segments, which reduces the frustration inherent to repeated playthroughs of a level. Additionally, the faster players are able to heal during a combat encounter, a lesser amount of mistakes will be committed when deciding if there is an appropriate timing window for the player to consume such an item without being vulnerable to enemy attacks. Therefore, the use of healing items in Dark Souls also requires the ability to predict enemy patterns and accurate decision making regarding appropriate timing windows.

% Suggestions
%   - Option to turn on more eastus uses per bonfire, possibly even infinite;
%   - Have automatic eastus uses when the player’s health drops below threshold.
While veteran players may be able to accurately recognize enemy behavioral patterns and attack animations and thus commit less mistakes regarding the usage timing of healing items, beginner players are often unable to perform such decisions at the critical point of combat encounters. This results in the frustration of being unable to properly sustain through combat encounters, which affects the learning curve. As a result, the authors suggest two approaches to alleviate the sustainment of player resources through the exploration of game segments, as well as the strategic use of healing items during combat encounters: first, an option to increase the number of \emph{Eastus Flasks} that can be use per each interaction with a \emph{bonfire}; and second, have an optional automatic system for the use of Eastus Flasks when the amount of player health points drops below a certain threshold.

% Problem: Punishment on Dying
%   - Two types of punishment: loss of souls (experience) and progression over explored environment;
%   - Punishment systems increase time spent playing;
%   - Knowledge of punishment on failure may stops players from trying;
%   - Players should be able to quickly return to point of failure and practice;
% Suggestions
%   - Toggle to turn off the mechanic of losing your souls when player dies.
%   - Option to restart fight instead of returning to bonfire.
The final difficulty factor identified in \citet{ONLINE_ApproachabilityFixDarkSouls} is relative to the punishment mechanics involved when a player dies. As previously discussed, punishment mechanics in video games are employed with the objective of increasing the sense of tension attached to the loss condition, which creates incentives for the player to perform actions which achieve success. Most punishment systems involve the loss of some type of progress by the player, which in consequence increases the time a player requires to complete the game.

If the losses in progression caused by failure are too significant and the chances of failing are too high, withdrawal is incentivized as the player might feel demotivated by the lack of confidence in their skills after considering the undesired effects of failure. Therefore, it is important to balance the loss progression from punishment mechanics with the expected frequency of failure by players at a given game segment.

In Dark Souls, two types of punishment occur on player defeat: the loss of all carried experience points (named "souls"); and the loss of progression over the explored environment before the player is able to reach the next checkpoint. Experience points are gathered by defeating enemies, and each time the player interacts with a bonfire (either by death or upon reaching a new bonfire) all previously eliminated enemies are re-instantiated -- which creates an opportunity for infinite gathering of experience points.

Therefore, it can be argued that the aversion from loss of experience points is mitigated by the fact that the player has access to an unlimited supply of experience points, if so desired. Additionally, lost experience points can be recovered at the location of player death until the player is defeated again, which creates incentive for players to repeatedly attempt to traverse through the same game sections to avoid the loss of experience points, while also accumulating previous experience points with the amount gathered during the traversal of the same, already known path.

However, this system might still cause frustration upon repeated deaths after the player defeats \emph{boss enemies}. In these difficult combat encounters, players are rewarded a significantly higher amount of experience points in comparison to defeating normal enemies. After defeating a \emph{boss enemy}, the player has to reach a bonfire before they can spend their experience points on improved attributes. After a \emph{boss fights}, the player will commonly have spent most of their sustainment resources, which means that traps or slight mistakes are likely cause their death. Consequently, players will repeat the traversal of the path until where the \emph{boss fight} occurred to recover their lost experience points, and might die again.

Therefore, the mechanics and systems that are supposed to mitigate the frustration from loss of experience points might also be contributing factors to the same loss of progression. To mitigate this issue, the author proposes an accessibility option for the removal of the loss of experience points upon death, which greatly reduces frustration for beginner players.

Regarding the loss of exploration progression, where players are required to successfully repeat long game segments until reaching the next checkpoint (or "bonfire"), players might become more accustomed to combat encounters which occur closer to their spawn point. Because of the incremental loss of sustainment resources over the traversal of environments, chances of player death are increased after each combat encounter, which means that encounters that are closer to the player spawn point will likely be repeated with a higher frequency in comparison to the encounters that are located closer to the next bonfire.

Therefore, the player has less chances to devise strategies and adapt to the characteristics of these later encounters. To mitigate this characteristic, the author proposes the creation of intermediary checkpoints, where players are able to re-spawn closer to the combat encounters where they were defeated -- which means that players have more opportunities to test out different strategies for a specific combat encounter, and consequently spend much less time to improve their execution.

% * Online: How challenging is Dark Souls and what makes it difficult?
% * ====================
% \cite{YT_HowChallengingIsDarkSouls}
We also explore some of the observations, perceptions and opinions of the Dark Souls player base to understand the general consensus of players regarding the difficulty for beginner players. To this objective, we perform an analysis of online videos created by members of the Dark Souls player base which promote the discussion of aspects of the level design, game mechanics and systems that contribute to the public consensus of Dark Souls as a difficult and challenging game.

In the discussion promoted in \citet{YT_HowChallengingIsDarkSouls}, we observe that one of the main aspects that can contribute to the impression of difficulty is related to the fact that the game does not provide in-game interface elements or systems which explicitly inform the player of which path should be traversed at any given point in the game. Therefore, beginner players will often be required to infer their objectives through an experimentation process where they explore multiple possible paths, and assess which paths can be successfully traversed given their current equipment, level and attributes.

This can result in beginner players continuously attempting to traverse paths with difficult or impossible progression given their current attributes, which causes frustration and demotivation. One of the most prevalent examples for this can be observed when players reach the bonfire for a game location named "Firelink Shrine", where players can either achieve progress by traversing the "Undead Burg" or incorrectly attempt to progress through the "Catacombs", which presents stronger enemies and will not provide progression of game story elements.

While some commercial games attempt to mitigate this issue by providing visual indicators as to the current objectives of a player and the path they should take given a game world state, we argue that part of the core appeal of Dark Souls is the process of exploration, experimentation and discovery. Therefore, the use of visual indicators could negatively influence the core experience and immersion. A crucial factor for the heavy atmosphere in Dark Souls is the feeling of loneliness. Considering this, we argue that a better solution would be to create level layouts that restricts player decisions or induces the player to traverse through the correct path in regards to story progression.

% * Online: Prepare to Die by Simple AI - Dark Souls and Difficulty | Design Dive
% * ====================
% \cite{YT_DarkSoulsSimpleAI}

% Lack of instruction
The discussion presented in \citet{YT_DarkSoulsSimpleAI} relates previous research on game difficulty to the overall perceptions of the Dark Souls player base, and elicits that the overall lack of guidance and instruction to players can affect the perceptual impression of AI opponents. Because of the design characteristic of failure used as a tool for learning and improvement, players often present a more cautious approach upon first interaction with a new enemy, as they recognize that one of the most important factors to guarantee survivability against an enemy is to understand and recognize their behavioral patterns. 

% Overt Strength - perceived intelligence
A trend regarding the attributes and parameters of AI opponents presented in Dark Souls is the overt strength of enemies in comparison to the attributes and equipment used by the player.  Players often engage in combat encounters with enemies where any attack that is not properly defended, dodged or parried against might cause a significant loss of health points or even death. Additionally, it is common for the amount of health points for such enemies to be much higher than that of the player or enemies from previous game segments.

% Unwieldy Mechanics
Another characteristic is observed relating to the ease of execution of mechanics by beginner players, where Dark Souls presents unwieldy mechanics that can be hard to become accustomed to, and thus severely affect the difficulty and learning curve depending on player choices.

% Long animations & stamina cost depending on weapon type
This can be observed when considering the attack animations of the multiple weapon types, where smaller weapons such as daggers, short swords and long swords present shorter attack animations and cost a lesser amount of Stamina. This reduces the time where the player is vulnerable by \emph{animation locking}, which renders the player unable to perform the dodge and block defensive actions after an attack.

Heavier weapons such as two-handed swords, large axes and mauls have longer animation times and stamina costs, making players animation-locked for a significantly higher duration, and thus vulnerable to attacks. The effect of equipment in the execution of game mechanics can also be observed with the \emph{parry} mechanic, where different types of shields can affect the timing window where attacks can be effectively parried. Additionally, the same effect can be observed when players equip different sets of armor, which modify the weight attribute and affect the speed of dodging animations.

% Attack collision with environment
The difficulty of combat mechanics can also be observed through the collision of attacks with environment geometry, where if a player engages in a combat encounter in constricted spaces, it is likely that their attacks might collide with walls. This event causes multiple consequences, including animation locking in a brief stagger animation, cancelling of attack animations that causes enemies to not be hit, and degradation of \emph{weapon durability}.

% Level design that punishes risky approaches
In conclusion, the work presented in \citet{YT_DarkSoulsSimpleAI} also analyzes level design, with specific focus on the placement of enemies and traps as to punish players which do not prioritize a careful approach. We can relate this to the previously discussed observations in section \ref{sec:level-design-dark-souls}, where we performed our analysis of level layouts and enemy placement. We observe that game designers often position enemies outside the player's field of view, as well as in strategic places that might cause instantaneous death such as the edges of a cliff. Designers also employ the use of certain enemies as a distraction that, when the player engages combat with, obfuscate the presence of another enemy that can perform a \emph{back-stab}.

Therefore, the level design in Dark Souls creates an incentive for players to carefully analyze, evaluate, plan and execute while minimizing the risks and memorizing the characteristics of paths taken, in order to avoid significant losses of resources or even instantaneous death. We argue that at least in the case of beginner players, we can alleviate the learning curve by eliminating the use of constricted spaces, traps and distractions based on their performance, causing beginner players to focus on learning basic combat mechanics and recognizing enemy patterns.

\section{Conclusions}

We conclude our analysis of Dark Souls by synthesizing a summary of the importance of its choice as an object of study, the game design characteristics which can be used as a reference for the implementation of a simplified version of the game, and the inherent issues with difficulty caused by design choices. We also propose solutions to the inherent problems of game design in Dark Souls, which are used as a basis to the choice of our adjustment policies and targets in section \ref{sec:adjustments}.

\subsection{Relevance}

% - Source of controversial discussions on difficulty & accessibility
As discussed earlier in this chapter, Dark Souls has been one of the most influential games when considering the discussion regarding difficulty, challenge and accessibility in games, as seen in \cite{ONLINE_GettingWrongDarkSouls}. Dark Souls was the source of many controversial innovations and design approaches involving level design, combat systems, and punishment of failure \cite{ONLINE_ToughLoveDarkSoulsDifficulty}. Over the last decade, Dark Souls was also the protagonist in the discussion of the newly popularized concept of approachability to beginner players, as seen in \cite{ONLINE_ApproachabilityFixDarkSouls}. Therefore, we argue that the use of Dark Souls as an object of study can be used as a platform to improve our understanding of the impact of difficulty in player experience.

% - Failure as part of learning process
Dark Souls was also one of the first games which addressed the use of failure as a part of the learning process by players, where the cycle of exploration, experimentation and adaptation created a methodology of iterative improvements to the player learning process. However, this methodology also proved to be a controversial approach, as many players report feeling frustrated and decide to give up on the game, resulting in a completion rate of approximately 26\% of the player base \cite{ONLINE_ApproachabilityFixDarkSouls}. Therefore, the use of Dark Souls as an object of study also provides us with opportunities to validate the use of adaptive systems to alleviate the learning process of beginner players, which in turn would reduce frustration and possibly reduce churn rates.

% - Approachability issues
As evidenced by our analysis of its difficulty factors, Dark Souls presents many issues regarding its approachability, which are a consequence of the inherent issues of its design decisions and a biased perception of players regarding its challenge level. Therefore, Dark Souls also provides us with many possibilities regarding adjustments of systems which have inherent approachability issues. In consequence, it is possible to use Dark Souls as a platform to N-dimensional dynamic difficulty approaches, where the difficulty of specific game elements can be adjusted considering isolated metrics of the player model.

\subsection{Core Design Aspects}

In this section, we provide a brief summary of the core game design aspects that were discussed earlier in this chapter, with the objective of identifying the minimum set of features and systems from the original game which can characterize a simplified version. The simplified version can then be used as a target object to the implementation of adaptive systems, with metrics and adjustments that consider the difficulty factors and inherent game design issues of Dark Souls to beginner players.

% Game mechanics
%   - Camera types: Orbital & Lock-On Cameras
Regarding the aspects involving camera implementations, we identified that Dark Souls employs the use of Third-Person perspective cameras through \emph{Orbital} and \emph{Lock-On} cameras. Orbital cameras perform rotational movement using the player character as a pivot, and are restricted to a fixed distance radius. Lock-On cameras are used to automatically frame the player and a target enemy, and to simplify the movement of the player character in relation to their enemy. Upon collision with the environment geometry, both camera types are brought closer to the player character.

%   - Attributes: Health, Stamina, Poise
%   - Regenerating stamina
%   - Stagger
Dark Souls implements three core resource attributes regarding the statuses of entities in combat: \emph{Health}, \emph{Stamina} and \emph{Poise}. Health is tied to the condition of failure, which is the death of the player character. Stamina is the resource used as a limiter to performing combat actions repeatedly such as attacking, dodging, blocking and parrying. Stamina is automatically regenerated, and the regeneration is briefly halted when any action results in a change of its value. Poise determines the resistance of the player to the \emph{Stagger} status, where the player might become briefly animation-locked after getting hit by one or multiple attacks.

%   - Actions: Attack
%   - Recovery time
During combat, the player can perform a looping sequence of \emph{attacks} which is only limited by their amount of stamina. Each attack has an attached \emph{stamina cost}, and if the player has no stamina they are unable to attack. Attack animations also have a specific timing window where they are able to \emph{register a hit}, and each attack has a specific \emph{hitbox} collider which is used for the detection of hits. At end of an attack sequence, attack animations force the player character into the \emph{recovery time} state, where the player is \emph{animation-locked} to the end of the attack animation and unable to perform any actions. Attack animations are defined by the weapons being used, where lighter weapons use faster animations and heavier weapons use slower and longer animations.

%   - Actions: Block
%   - Actions: Parry
The player can also hold their weapon or shield to \emph{block} attacks, which reduces the rate of stamina regeneration by half. Blocking an attack has a stamina cost, and if the player has no stamina when performing a block they are automatically staggered. Each attack has a different stamina cost to block, with some enemies performing \emph{special attacks} that dramatically reduce player stamina. Additionally, players might also attempt to parry an attack, which completely negates the effects of player attacks and also staggers the target enemy if accurately executed. To correctly execute a parry, players must perform the action at a moment in which the \emph{parry timing window} matches the hit registering window of enemy attacks. If the player incorrectly performs the parry, they are left vulnerable and will suffer the full effects of the attack. Parry also has an attached stamina cost, and can not be performed if the player has no stamina left. 

%   - Actions: Dodge
Players can also attempt to dodge attacks, which serves as both an attack avoidance and a repositioning mechanic. To be able to dodge, the player is required to use a significant amount of stamina. If the player has no stamina, they are unable to perform a dodge. During a brief timing window at the beginning of the dodge animation the player is considered invincible to attacks, meaning that the effects of attacks from enemies that would register a hit are completely negated. The game presents different dodging animations depending on the armor type being used by the player, and each animation has a specific invincibility window.

% Artificial Intelligence
%   - Simplistic behaviors
%   - Stand idle or patrol until sensors detect player
%   - Attack patterns which encourage pattern recognition
%   - Attributes stronger than player
Regarding the implementation of AI agents in Dark Souls, most enemies present simplistic behaviors that perform a limited set of actions without considering many contextual aspects of combat. By default, most of the enemies will stand idle at a specific position or walk through a small set of waypoints until their sensors detect the presence of the player, at which point enemies will chase the player until close enough to attack. The exception is with the "archer" type of enemies, which simply attack the player from a distance while standing idle after detecting their presence. Enemies perform a limited set of attack sequences that encourage memorization and pattern recognition by the player to appropriately avoid. Most of the enemies present a fair challenge when considering the strength of their attributes in comparison to that of the player character, but some enemy types can present much higher health points or damage dealt to the player, which increases the level of challenge and encourages players to take less risks in combat.

% Level design
%   - Encourages memorization
%   - Open or constricted environments
%   - Placement of enemies & traps outside of player FOV
%   - Archer enemies that might be difficult to reach
Regarding the level design in Dark Souls, a variety of environments are presented with different characteristics. Open environments are more prevalent in areas where \emph{boss fights} take place, while most of the game occurs in constricted environments where player movement is limited. The game also performs heavy employment of enemies and traps in strategical positions that can be outside of the player field of view, which might catch an unsuspecting player off-guard and result in their death. Therefore, the game encourages memorization of the environments and the minimization of risks through a careful approach. In combat encounters, the game often positions ranged enemies in strategical advantage points which are difficult to reach, so the player has to traverse through long sections of the environment to be able to eliminate such enemies.  

% Aesthetics
%   - Dark fantasy setting
%   - Undead enemies
%   - Realistic, slow attack animations
%   - Feeling of unease
Regarding the aesthetics of Dark Souls, the game presents an environment in a \emph{Dark Fantasy} setting, where most enemies are undead or supernatural entities. The visual elements and audio design cause a general feeling of unease, which goes in accordance with the narrative aspects and the overlying theme of loneliness of a lost soul traversing through a ruined world. The animations for the player character and most humanoid entities are slow and heavy, providing a sense of weight and realism in combat.

\subsection{Difficulty Issues \& Proposed Solutions}

% DARK SOULS DIFFICULTY FACTORS
% ================================
% Lack of guidance & instructions
% Level design that punishes risky approaches
% Ambiguous attack animations
% Unwieldy mechanics (parrying, attack collision with environment, attack animations dependant on weapons)
% Death punishment mechanics (loss of souls, sparse checkpoints)
% Limited healing amount and animation locking
% Overt Enemy Strength
% Perceptual difficulty

Based on the detrimental factors to approachability analyzed in section \ref{sec:detrimental-factors-approachability}, we can synthesize a summary of the difficulty factors that are related to the design aspects of Dark Souls:

\begin{itemize}
    \item{\emph{Lack of guidance and instructions}, where the game provides no visual indicators as which environment paths the player should prioritize to progress in the game;}
    \item {\emph{Level design that punishes risky approaches}, where the game employs several traps, strategically positioned enemies or thin platforms that might cause player death if not addressed carefully;}
    \item {\emph{Ambiguous enemy behavioral patterns}, where movement and attacks performed by enemies might not be properly assessed by players, thus causing a player to perform an incorrect defensive action;}
    \item {\emph{Unwieldy mechanics}, where the execution of actions is affected by equipment, attributes and environment, such as attack animations tied to weapon types, attacks being obstructed by walls, shields having different parry timing windows and armors modifying the dodge animation speed;}
    \item {\emph{Punishment mechanics}, where the game punishes failure with progression loss such as the loss of experience points or explored environment due to sparse checkpoints;}
    \item {\emph{Limited sustainment resources}, where player progress is slowed by a limited amount of healing items that can be used before reaching the next checkpoint;}
    \item {\emph{Animation locking}, where the player becomes vulnerable when performing actions such as consuming an healing item or performing an attack with a heavy weapon;}
    \item {\emph{Overt enemy strength}, where enemies will deal much more damage and present much more health in comparison to player attributes;}
    \item {\emph{Combat Speed}, where a player might be unable to appropriately assess the type of attack being performed by enemies or execute the appropriate response in time to avoid being hit;}
    \item {\emph{Perceptual difficulty}, where the visual appearance or attributes of an enemy might create an incorrect perception of an AI opponent presenting more challenging behavior than the realistic nature of their algorithms.}
\end{itemize}

In the scope of this work, and because of restrictions in development assets and time for the creation of the proposed implementation, we choose to perform the following difficulty adjustments that address the some of the issues with design aspects that cause inherent difficulty factors to the design of Dark Souls: 

\begin{itemize}
    \item {\emph{Dynamic level layouts}, which can alleviate the difficulty inherent in the lack of guidance, or the difficulty of combat when engaging multiple enemies in constricted environments;}
    \item {\emph{Dynamic enemies placement}, which can alleviate the difficulty of specific combat encounters when the player faces a combination of enemy types;}
    \item {\emph{Dynamic enemy attack animations and visual indicators}, which alleviate the issues inherent with ambiguous attack animations performed by enemies;}
    \item {\emph{Dynamic enemy and group behaviors}, which can alleviate the requirements of quick pattern recognition and player decision-making with ambiguous behavioral patterns;}
    \item {\emph{Dynamic positioning of checkpoints}, which can alleviate the loss of exploration progression performed by a player that is unable to overcome a combat encounter before reaching the next checkpoint;}
    \item {\emph{Dynamic Game Speed}, which alleviates the issues with players being unable to execute the appropriate defensive response in time to avoid enemy attacks.}
\end{itemize}