\chapter{ANALYSIS OF OBJECT OF STUDY}
\label{ch:analysis-object-study}
\label{sec:analysis-dark-souls}

In this section, we analyze the Game Design of our object of study to define the essential aspects required to implement a replica of its core elements and experience. We argue that understanding the Game Design of \emph{Dark Souls} is crucial to the interpretation of difficulty in games. While \emph{Dark Souls} presents a wide variety of mechanics and interactions, only the basic rules required for the implementation of the combat system will be described.

% ============================================================================
% ============================================================================
% ============================================================================

\section{Motivation of Choice}

Dark Souls has successfully pushed the extents of challenge in Action RPGs to their limit, providing combat with a simple yet carefully designed set of core mechanics. However, the greatest capability of the \emph{Dark Souls} series, the design of its difficulty, is for some an unwelcome trait. While having enemies with simple pattern-based AI, the game punishes players heavily for making the slightest mistakes. For this reason, \emph{Dark Souls} is often mentioned as a reference to difficulty in games \cite{URL_ExploringDesignOfDarkSouls}.

The player does not fail in \emph{Dark Souls} by dying, but by giving up on finishing the game. The game design creates a cycle of experimenting, dying, learning, adapting and only then progressing through the levels. Therefore death is not considered failure, but simply the price paid for knowledge. Because of this type of punishing yet rewarding design, the special edition of the first title in the \emph{Dark Souls} series yields the subtitle "Prepare to Die Edition".

While the average action game might reward the player for beating entire armies of enemies on his own, \emph{Dark Souls} can be a rewarding experience for the player even when basic encounters are surpassed, such as surviving a battle with a single enemy. When adventuring through the tight corridors of a dungeon or dark underground caves, the player faces unknown dangers and must fight their way out. Even the weakest enemies can surprise and kill an unsuspecting player. When facing the strong opponents, dealing the finishing blow in an intense fight means the player overcame a challenge.

Understanding the design of Dark Souls is, therefore, a key component to understanding the relationship of challenge and reward.

% ============================================================================
% ============================================================================
% ============================================================================

\section{Overview of Game Mechanics}

\emph{Dark Souls} is for the most part presented in a Third Person Camera perspective, with the player character positioned close to the center of the screen. The Third Person perspective is distinguished by the visibility of the player character's model and actions and a slight view of their surroundings.

In comparison to First Person Camera Systems, and considering only mechanical gameplay aspects, Third Person Cameras are better suited to action games with platforming sections. In these sections, the player must have full control and awareness over the character position to avoid falling to their death. Third person perspectives also succeed in close-combat games by providing a broad view when the player deals with multiple enemies simultaneously.

However, as seen in \cite{BOOK_LevelUpTheGuideToGreat}, Third Person Cameras often have a problem of visibility in regards to the player character in tight or confined spaces of the game world, such as corridors and small rooms. The camera must be confined inside the game level boundaries to prevent inappropriate viewing of elements of the environment, such as  out-of-bounds artifacts.

In a Orbital Third Person Camera System, character movement is based upon the relationship of character position to the camera orientation \cite{BOOK_RealTimeCameras}. If the player moves forward, the character moves towards the direction the camera is facing. In Dark Souls, moving sideways also causes the camera to slightly rotate towards the direction of the movement. Therefore, if the player constantly moves horizontally without adjusting the camera, the character will move in a circle.

In situations where the camera position is invalid, the camera should reposition inside the playable bounds of the environment whenever the intended default position of the camera would be out of the limits or inside an object. According to \cite{BOOK_RealTimeCameras}, there is no standard solution to this issue. In Dark Souls, the adopted solution is to simply pull the camera closer to the player character. The relative size of the player in screen is then greatly increased and might disrupt the visualization of important gameplay elements. 

Environment elements such as columns, rocks or even non-playable characters might impair the visualization of the player character and non-player characters, thus hindering the combat capabilities of the player. In response, Third Person games with close-quarters combat often present the "Lock-On" camera as an alternative to the default.

A View Locking Camera System, such as the Lock-On Camera, overrides the orientation of a Third Person Camera by locking the orientation of the view to a "lock target". Instead of centering on the player character, the Lock-On Camera focuses on another object or position such as a non-playable character, while still offsetting the position of the player character. This allows the visualization of the two essential elements in a combat encounter: the player character and their enemy.

In the View Locking Camera System, the player movement method changes to a circular strafe relative to the lock target position. Moving sideways results in an arc-shaped motion around the target. Forward and backwards movement transposes the player character closer to or away from the target.

The Lock-On Camera is ideal for single-target combat situations, as the player can avoid enemy attacks sideways while still keeping a desirable distance to counterattack. Area of Effect attacks which affect any entity in a circular area around the enemy can also be easily avoided by moving away from the enemy. In the case of Dark Souls, the player character orientation is updated to constantly face the target, thus simplifying the process of directing attacks at enemies. Since the player character is constantly facing the direction of the target, successfully registering the attack simply requires the player to position their character in range of the target.

Dark Souls is an Action RPG with heavy focus on close-quarters combat. In contrast to other games in the genre, the combat does not present itself as fast-paced or dynamic. Instead, the game focuses on realism and slow but strategic movements. According to \cite{BOOK_DarkSoulsBeyondTheGrave}, \emph{Dark Souls} respects its predecessors such as \emph{King's Field} with a feeling of weight and impact on hits, emphasizing the necessity of players to protect themselves behind a shield.

The controls in any game of the \emph{Souls} series operate in similar fashion: each of the character's arms are controlled by separate buttons, represented by the left and right shoulder pad in a Joystick. The player can wield different one-handed equipments in each arm, or a two-handed equipment in both arms. The character will perform actions for an arm based on which equipment is equipped, such as slashing with a sword or defending with a shield. If a two-handed weapon is equipped, using the right shoulder pad button will attack and the left will defend. The most common configuration is to have a shield in the left hand, and a weapon in the right.

During play-through, the player will come across a wide variety of weapons with different attacks, but the same core functionality. Each attack requires a certain amount of \emph{Stamina}, deals a certain amount of \emph{Damage} and has a \emph{Recover Time} after the action. Certain attacks also have the chance of \emph{Staggering} an enemy. These core concepts are strategic factors the player must consider before taking action. 

\begin{itemize}

\item \emph{Damage}: the amount of health a character loses upon getting struck by an attack. This amount can be reduced or fully negated by blocking or dodging incoming attacks.

\item \emph{Health}: an attribute representing the overall physical state of a character. Numerically, it determines how much Damage a character can sustain before being destroyed.

\item \emph{Stamina}: an attribute that represents the character tiredness. Numerically, it determines the ability to perform actions in combat. Attacking, dodging and even shielding oneself against enemy attacks all have a \emph{Stamina} cost. Although \emph{Stamina} is a finite resource, it replenishes passively after the player stops performing combat actions.

\item \emph{Recover Time}: the amount of frames the player is unable to do any actions and is vulnerable to attacks after their own attack has taken place. This attribute is used to create a feeling of weight in attacks, and can be used by the player to punish slow enemies after missing an attack.

\item \emph{Stagger}: the condition where a character is unable to perform any actions and is vulnerable to attacks after getting struck by a powerful attack or a quick succession of attacks. In \emph{Dark Souls}, this condition is especially cruel to the player, since after losing a significant amount of health the player is still susceptible to a follow-up attacks, without the ability to defend their self.

\end{itemize}

In general, Heavy attacks are slower and cost the most Stamina, but deal the highest damage and \emph{Stagger} the enemy. Light attacks are faster and can be linked for a quick succession of attacks, but deal individually less damage while having a low chance of causing \emph{Stagger}.

As means of defense, a player can Block, Dodge, Parry or simply move away from an attack. Blocking requires the player to have their shield lifted at the time of the attack. Each block costs an amount of Stamina based on the power of the attack and the character's resistances. If an enemy attack is too powerful and the character does not have the required Stamina, they may still enter a Stagger condition. Out of all the defensive actions, Blocking is the safest and least skill dependent, since it will only fail if the attack hits the back of a character. While blocking, the player will replenish Stamina at a reduced rate.

Dodging can be performed by rolling in the right direction with precise timing. While costing Stamina, succeeding in this action completely negates the damage of an attack, ignoring how powerful it is. Upon failing, the character will take full damage. Thus, this action should be performed with care and heavily depends on the player reflexes.

Similarly to Dodging, Parrying can be performed with precise timing to deflect an enemy attack and destabilize the enemy. The player can then perform a riposte attack, a powerful move that adds a considerable amount of damage. Failing a Parry will cause the player to take full damage of the attack. Parrying is harder than dodging, since time window to perform the action is considerably smaller. Thus, Parrying is a high-risk high-reward action, and can also be considered an offensive move.

However, Blocking, Parrying and Dodging require the usage of Stamina, a valuable resource which is also required to perform offensive moves. If a player only blocks incoming attacks, their Stamina will not replenish quickly enough to sustain attacks indefinitely. Moreover, if a player faces multiple enemies, blocking can sustain even less attacks, and quickly deplete the player Stamina. Dodging and Parrying are timing dependent, and thus prone to mistakes and heavy punishment. If players repeatedly try to perform the same defensive action, they will often find themselves cornered, without Stamina and vulnerable to fatal blows. Experienced players have the knowledge of when they should perform none of these actions, instead simple moving away and creating space for favorable counterattack opportunities.

% ============================================================================
% ============================================================================
% ============================================================================

\section{Artificial Intelligence}

Upon the research conducted in the creation of this work, no official information on the implementation of the AI in Dark Souls was discovered. According to \citeonline{YT_DarkSoulsSimpleAI}, some sources indicate a \emph{Hierarchical Task Planning Network System}, but an analysis of the in-game enemy actions provides no support to this claim. However, by multiple tests and playthroughs, it was possible to reverse engineer the behavior of enemies and suggest a replication formulae in the implementation of this work.

Non-player characters often share similar behavioral patterns in Dark Souls. Whi\hyp{}le the actions of a humanoid NPC might differ from a quadruped, their overall behavior upon player presence is the same. The NPC stands idle until receiving interaction from their sensors, such as the player stepping into their line of sight. At that point, the enemy will either attack at range, if using a bow or spell, or rush towards the player and attack in close-quarters.

Enemies wearing a shield will commonly attempt to defend themselves when the player repeatedly attacks. The defense stance can be punished by the player when they perform a kick or by attacking from behind. In other occasions, a low health enemy might evade attacks or even heal itself. Most of these strategies are predictable, even when there is a variability in the patterns used.

Ultimately, the AI of Dark Souls is simplistic, but sufficient for the objectives proposed by the developer. The predictability of this system can be considered a favorable factor for a player learning how to overcome an enemy. This simplicity is not perceivable by the player in the first playthrough, since the player will struggle with the challenging aspects of the level design. In addition, Dark Souls present its enemies as being overtly strong in comparison to the player. A slight mistake might cause the player to endure punishing amounts of damage, thus creating the illusion of a difficult opponent.

The perception of strength in Dark Souls's enemies carries similarities to the development of the original \emph{Halo} by Bungie. According to \citeonline{URL_IllusionOfIntelligence}, players perceived the smartness of an enemy based on their endurance and damage dealt to the player. Simply increasing the Health points of an enemy enhanced the first perception of enemies intelligence for playtesters. This perception of toughness could shadow the lack of intelligence of an enemy in a first playthrough. However, as the player repeatedly faces the same enemy upon countless deaths, this illusion is gradually faded.

% ============================================================================
% ============================================================================
% ============================================================================

\section{Level Design}

According to \cite{BOOK_LevelDesignConcept}, Level Design can be defined as an interpretation of Game Design. The Level Designer must understand the rules of a game and determine how a player is confronted by them. It can be argued that Game Design represents the theoretical part of a game, whereas Level Design applies it in practice.

Level Design determines the layout of a location, the placement of enemies, the gameplay objects and the environmental hazards. In a sense, Level Design can be seen as a means of expressing Game Design through an exploration narrative. Therefore, a game developer must consider what experiences the player is supposed to face in a section before tackling on its design.

In the case of Dark Souls, the world takes place in an open, interconnected and vertically stacked map layout. When projecting this layout on a two-dimensional chart, it resembles the format of a spiral. This type of map layout is radically different than other games in the genre such as \emph{The Elder Scrolls V: Skyrim} \footnote{The Elder Scrolls V: Skyrim (Bethesda Game Studios, 2011). Computer Game. Microsoft Windows.}, which presents  dungeon maps as horizontally spaced and linear paths.

By journeying through the world, the player will come across castles, dungeons, caves, and fortresses. It is more frequent to find oneself in small passageways than open areas, the latter being used more often in \emph{boss fights}. The need of constantly turning left and right, ascending stairs and descending through dark passages gives the developers numerous places to hide enemies and traps in. Thus, the player often faces threats such as being assaulted by an unseen enemy, getting hit by a trap or even falling to their death.

Each and every enemy the player faces has the potential to generate a difficult encounter. An enemy that is considered weak can still take away a considerable amount of Health from the player. However, enemies are commonly vulnerable after performing an attack, and thus open for an instant kill. Therefore, the player has the possibility of optimizing their play style by spotting enemies ahead of time, planning on how to exploit weaknesses and only then performing the action.

The strength of a \emph{Souls}-like game isn't just its combat. When every little bit of a level can be threatening, it encourages the player to experience the atmosphere and story of the world. The careful placement of enemies, traps and pitfalls in creative fashion constitutes the core of \emph{Dark Souls} Level Design.

A Dark Souls player is encouraged to memorize whole map sections to progress withstanding minimal damage. The constant feeling of danger increases the tension and maintains the player aware of their situation to survive. According to \citeonline{YT_EvolutionOfDarkSoulsLevelDesign}, the Level Design in  Dark Souls is the main contributor to the player immersion in the game's narrative.

However, the pitfalls and traps of Dark Souls can be spotted ahead of time if the player decides to maintain a slow pacing an pay attention to their surroundings. This characteristic makes the seemingly unfair encounters beatable. Groups can be separated into smaller sizes by luring enemies one by one. The environment traps can be used against the enemies, if the player lures the target to the area of effect. Enemies can be pulled from their territory into a safer and player-controlled position.

Spiral level designs can also feature alternate routes and shortcuts. Since the player is constantly facing the danger of losing Experience Points upon dying, the Spiral layout provides the player with shorter paths and less risk when returning to safe zones. In the same philosophy, the game will often contain shortcuts from safe zones to later attained areas. These shortcuts must be unlocked by reaching a certain location and performing an action, such as finding the key to a locked gate. This Level Design technique is commonly known as \emph{gating} \cite{BOOK_LevelUpTheGuideToGreat}.

% ============================================================================
% ============================================================================
% ============================================================================

\section{The "Pain Points" of Dark Souls (TO DO)}
\label{sec:pain-points-dark-souls}