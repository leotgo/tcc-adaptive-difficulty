% Abstract in the same language as the rest of the document
\begin{abstract}
    The Video Games Industry has learned through trial and error how to make digital games appeal to broad audiences. Qualities such as pleasing aesthetics, appealing story and concise playability are requirements for player engagement. However, there is a tendency for making games easier to support entry-level players. While this is a respectful approach to embrace casual players, it might bore skillful players which dedicate more time to gaming. One interesting solution to this problem is the concept of "adaptive difficulty". Instead of reducing difficulty curves to finite menu options such as "Easy" and "Hard", difficulty is treated as a set of in-game continuous variables in multiple layers, each pertaining to a specific set of the game's rules. The game monitors player actions and their results to dynamically adjust in-game parameters and tailor the difficulty of game systems to the specific needs of a player. Therefore, the player does not have to manually input their desired difficulty level, and the game smoothly adapts to the player's profile. We review multiple dynamic difficulty methods proposed in commercial games and in prior academic work, and present an implementation of player-centric adaptive technology in games. We evaluate the benefits of personalizing the game based on user preferences and performance, using the \emph{Dark Souls} game series as an object of study. We implement a subset of the \emph{Dark Souls} game mechanics with a simplified version the features presented in the original game. The customization methods include dynamic and subtle gameplay changes, a Dynamic Difficulty Adjustment system and a player model based approach for performance tracking.
\end{abstract}

% Second language abstract
% Parameters must include the title and keywords
% in the other language, separated by commas
\begin{englishabstract}{Uma Implementação de Sistemas de Dificuldade Adaptativa para Jogos Desafiantes}{Dificuldade adaptativa. adaptação dinâmica de conteúdo. criação de modelo de jogador. interação humano-computador.}
    A indústria de video games aprendeu, por tentativa e erro, como fazer jogos serem atraentes para um público grande. Estética agradável, uma história envolvente e jogabilidade concisa são características necessárias para engajar o jogador. No entanto, existe uma tendência em fazer com que jogos não sejam desafiantes para que sejam acessíveis a jogadores novatos. Enquanto que esta solução respeita a as necessidades de jogadores casuais, jogadores mais habilidosos que dedicam mais tempo a jogos podem se sentir entediados e desmotivados a jogar. Uma solução para este problema é o conceito de "dificuldade adaptativa". Ao invés de tratar dificuldade como um número finito de opções de menu nomeadas "Fácil" e "Difícil", a dificuldade é tratada como um conjunto de variáveis contínuas em múltiplas camadas, cada uma representando um conjunto de regras do jogo. O jogo monitora as ações do jogador e seus resultados, de forma a ajustar os parâmetros do jogo para se ajustar ao perfil de dificuldade do jogador. Portanto, neste trabalho apresentamos uma implementação para adaptação dinâmica de conteúdo em jogos centrada no jogador, e avaliamos os benefícios causados pela adaptatividade baseada em preferências e performance do usuário. O objeto de estudo é uma implementação de jogo inspirada no título \emph{Dark Souls}, com um conjunto reduzido e simplificado de mecânicas do jogo original. Os métodos de personalização incluem mudanças sutis na jogabilidade baseadas em preferência, um sistema de Ajuste de Dificuldade Dinâmico e mudanças no ambiente baseadas em estado de jogo.
\end{englishabstract}