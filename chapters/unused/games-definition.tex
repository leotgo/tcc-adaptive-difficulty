\chapter {Definition of Games, Experience and Difficulty}

\section{A Systematic Definition of Games}

It is possible to define digital games purely as being interactive systems, similar to some types of computer software \cite{ARTICLE_FromUsabilityToPlayability}. Whereas computer software is commonly developed with the objective of the user executing a set of tasks assisted by an automated system, video games are used for leisure purposes. Thus, it is necessary to understand what aspects constitute a game and how it differentiates from other interactive systems before defining player experience.

Perhaps the most precise definition of games in recent academic literature is discussed in \cite{ARTICLE_TheGameThePlayerTheWorld}. Their work claims the existence of a standard for creating games that has been constant for thousands of years. While many other definitions exist, we chose the definition by \citeauthor{ARTICLE_TheGameThePlayerTheWorld} because it compares and formulates over the definitions of various authors from the history of game research. Their work also proposes specific components to their definition, and discusses the relation of game systems to their players. Thus, \citet{ARTICLE_TheGameThePlayerTheWorld} proposes the following definition of \emph{games}:

\begin{quotation}
"A game is a rule-based formal system with a variable and quantifiable outcome, where different outcomes are assigned different values, the player exerts effort in order to influence the outcome, the player feels attached to the outcome, and the consequences of the activity are optional and negotiable.."
\end{quotation}

This definition is not restricted to any types of medium. Rather, it can be applied to \emph{analog games}, digital games, sports and even gambling. The author explains that this definition is characterized by six components: \emph{Rules}, \emph{Variable and Quantifiable outcome}, \emph{Valorization of outcome}, \emph{Player Effort}, \emph{Attachment of the player to the outcome} and \emph{Negotiable consequences}.

Games have \emph{Rules} that must be sufficiently well defined so that they are clearly understood and agreed upon by all participants. In the case of a non-electronic game, disagreement on the rules would result in the game being stopped and discussed over. In the case of digital games where the player interacts directly to the system, the developer should make the rules unambiguous and clear enough for a player to understand by their self.

The rules of a game must provide \emph{Variable outcomes}. For example, making different choices should result in different outcomes for a player. The author also argues that if player actions do not influence the outcome of a game, then the game can not be classified as a game \emph{activity}. The game must have \emph{Quantifiable outcomes} beyond the uncertainty of discussion. If a player performs an action, it should have a clear and predefined outcome such as winning or losing points.

Some of the possible outcomes of a game are better than others, thus creating the \emph{Valorization of the outcome}. For instance, in a multiplayer game the difference in the positivity of outcomes creates the conflict between players. Players will value most the actions which produce the better outcomes for their self interest. The author also states that there is a tendency that positive outcomes are harder to reach than negative outcomes, this being the cause of difficulty or challenge. 

The concept of \emph{Player effort} emphasizes that player actions must influence the outcomes of a game. The author argues that the effort put in performing these actions tends to lead to an attachment of the player to the outcome, since the player would be in part responsible for the results. The \emph{Attachment of the player to the outcome} is defined as a psychologycal feature of the game activity by which the player is affected by the outcome. A player may feel happy if they win, or unhappy on loss. However, the attachment to outcome is not just related to the effort since the player can still feel happy for winning games by pure chance. The author further argues that the attachment to outcome depends on a player's attitude towards a game and is part of the game contract.

The author defines \emph{Negotiable consequences} by discussing that games can optionally be assigned real-life consequences. The assignment can be negotiated on a per-play, per-location or per-person basis. One example for this would be gambling games where the player must bet a minimal amount before playing. The author emphasizes that some games may also have non-negotiable consequences, such as sports where a player might have an injury.

Defining games as systems is useful in explaining the necessity of choice and consequence. It also discusses one possible argument for the motivational pull of games: the \emph{Attachment to outcome}.  While the outcomes of game systems can be motivating factors to the act of play, we argue that they are not sufficient in fully explaining player experience. For instance, \emph{horror games}[FN] might cause the feeling of uneasiness and discomfort. These types of games can also be analyzed in terms of rule-based formal systems, but one could argue that this is not the nature of their motivational pull. The definition of such games implies that the player will inevitably experience negative feelings and outcomes. To understand how negative feelings and outcomes in games can contribute to an overall positive experience, we must study what constitutes \emph{Player Experience}.

\section{The Many Types of Difficulty}

Traditionally, Game Design employs difficulty as a static experience \cite{article_casefordynamicdifficulty}, optionally with a fixed set of modifiers. In the Arcade Era of games, titles such as \emph{Pac-Man} \footnote{Pac-Man (NAMCO, 1980). Video Game. Arcade.} and \emph{Donkey Kong} \footnote{Donkey Kong (Nintendo, 1981). Video Game. Arcade.} started with a fixed difficulty, which increased each time the player completed a level. In the more recent Console Era of games, titles such as \emph{God of War 3} \footnote{God of War 3 (Sony, 2010). Playstation 3.} and \emph{NieR:Automata} \footnote{NieR:Automata (Platinum Games, 2017). Microsoft Windows, Playstation 4, Xbox One.} offer multiple difficulty presets. Before starting the game, the player chooses between Easy, Normal and Hard modes. The initial difficulty of any mode slightly increases with story progression by presenting new enemies, obstacles and boss fights.

Recent games such as \emph{Dragon Quest XI} \footnote{Dragon Quest XI (Square Enix, 2017). Microsoft Windows, Playstation 4, Xbox One.} provide further customization of difficulty with selective features. Rather than choosing between predefined difficulty presets, the player will toggle one or multiple settings such as reduced experience gained from fights, tougher enemies, not being able to flee from battle and disabling the usage of protective armour. The game does not reward the player for completing the game with these settings. However, the player still achieves a better sense of accomplishment.

The advance of selective difficulty is a step forward on appealing to the specific preferences of players. It also enables a replayability factor for experienced players, where they can attempt a harder challenge after accumulating enough experience on the mechanics of the game. 

Another solution to difficulty is the Adaptive approach, which proposes the runtime adaptation of in-game systems to the skill level of any player. In contrast to selective difficulty systems, adaptive systems are able to work in a real-time, transparent and non-intrusive way \cite{PHD_DynamicDifficultyAdjustment}. They do not require direct player interaction with the difficulty systems, and instead rely on statistical gameplay data. In this system, a novice player is not required to select a difficulty level. The system detects which aspects of gameplay the player struggles with and adapts accordingly.